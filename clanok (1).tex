% Metódy inžinierskej práce

\documentclass[10pt,twoside,slovak,a4paper]{coursepaper}
\usepackage[slovak]{babel}
%\usepackage[T1]{fontenc}
\usepackage[IL2]{fontenc} % lepšia sadzba písmena Ľ než v T1
\usepackage[utf8]{inputenc}
\usepackage{graphicx}
\usepackage{url} % príkaz \url na formátovanie URL
\usepackage{hyperref} % odkazy v texte budú aktívne (pri niektorých triedach dokumentov spôsobuje posun textu)

\usepackage{cite}
%\usepackage{times}

\pagestyle{headings}

\title{YouTube and Movie Recommendation System Using Machine Learning\thanks{Semestrálny projekt v predmete Metódy inžinierskej práce, ak. rok 2024/25, vedenie: Kataieva Eugenia
}} % meno a priezvisko vyučujúceho na cvičeniach

\author{Samuel Máni\\[2pt]
	{\small Slovenská technická univerzita v Bratislave}\\
	{\small Fakulta informatiky a informačných technológií}\\
	{\small \texttt{xmani@stuba.sk}}
	}

\date{\small 26. september 2024} % upravte



\begin{document}

\maketitle
\begin{abstract}
	Systémy na odporúčanie videí sa stali dôležitou súčasťou internetových platforiem, ako sú Youtube a Netflix. Ich hlavným cieľom je zvýšiť angažovanosť a spokojnosť používateľov. Tieto systémy využívajú pokročilé techniky strojového učenia na predpovedanie a navrhovanie obsahu, ktorý zodpovedá individuálnym preferenciám používateľov. Analýzou správania používateľov, historických údajov a kontextovej informácie môžu odporúčacie systémy prispôsobiť odporúčanie obsahu tak, aby maximalizovali relevantnosť.
Ako to vlastne funguje? Táto štúdia skúma využitie algoritmov strojového učenia v kontexte odporúčacích systémov pre filmy(Netflix) a videá(Youtube). Systém identifikuje potencionálny obsah  na základe záujmov a interakcií používateľa, ako je história sledovania, hodnotenie obsahu, vyhľadávacie dopyty a mnoho ďalších. Následne systém tieto zozbierané dáta priradí k takzvaným „kandidátom“ na základe systému hodnotenia. Čím vyššie hodnotenie obsah obdržal tým vyššie bude v tabuľke kandidátov a teda má väčšiu pravdepodobnosť že práve on sa ukáže používateľovi. Použili sme takzvaný KNN algoritmus ako základný model strojového učenia a získali sme presnosť až 98 percent. Tento algoritmus efektívne zachytáva podobnosť medzi pozretým obsahom, čo umožňuje presné odporúčania.
Tieto zistenia dokazujú veľmi vysokú účinnosť strojového učenia pri odporúčaní obsahu na internetových platformách, konkrétne Youtube a Netflix. Využitím údajov o používateľoch a pokročilých algoritmov môžu platformy poskytovať vysoko relevantný obsah pre ich používateľov.

Zdroj: https://ieeexplore.ieee.org/abstract/document/10084999

\ldots
\end{abstract}



\section{Úvod}

Motivujte čitateľa a vysvetlite, o čom píšete. Úvod sa väčšinou nedelí na časti.

\section{Uvod2}
Uveďte explicitne štruktúru článku. Tu je nejaký príklad.
Základný problém, ktorý bol naznačený v úvode, je podrobnejšie vysvetlený v časti~\ref{nejaka}.
Dôležité súvislosti sú uvedené v častiach~\ref{dolezita} a~\ref{dolezitejsia}.
Záverečné poznámky prináša časť~

hello world\ref{zaver}.



\section{Nejaká časť} \label{nejaka}

Z obr.~\ref{f:rozhod} je všetko jasné. 

\begin{figure*}[tbh]
\centering
%\includegraphics[scale=1.0]{diagram.pdf}
Aj text môže byť prezentovaný ako obrázok. Stane sa z neho označný plávajúci objekt. Po vytvorení diagramu zrušte znak \texttt{\%} pred príkazom \verb|\includegraphics| označte tento riadok ako komentár (tiež pomocou znaku \texttt{\%}).
\caption{Rozhodujúci argument.}
\label{f:rozhod}
\end{figure*}



\section{Iná časť} \label{ina}

Základným problémom je teda\ldots{} Najprv sa pozrieme na nejaké vysvetlenie (časť~\ref{ina:nejake}), a potom na ešte nejaké (časť~\ref{ina:nejake}).\footnote{Niekedy môžete potrebovať aj poznámku pod čiarou.}

Môže sa zdať, že problém vlastne nejestvuje\cite{Coplien:MPD}, ale bolo dokázané, že to tak nie je~\cite{Czarnecki:Staged, Czarnecki:Progress}. Napriek tomu, aj dnes na webe narazíme na všelijaké pochybné názory\cite{PLP-Framework}. Dôležité veci možno \emph{zdôrazniť kurzívou}.


\subsection{Nejaké vysvetlenie} \label{ina:nejake}

Niekedy treba uviesť zoznam:

\begin{itemize}
\item jedna vec
\item druhá vec
	\begin{itemize}
	\item x
	\item y
	\end{itemize}
\end{itemize}

Ten istý zoznam, len číslovaný:

\begin{enumerate}
\item jedna vec
\item druhá vec
	\begin{enumerate}
	\item x
	\item y
	\end{enumerate}
\end{enumerate}


\subsection{Ešte nejaké vysvetlenie} \label{ina:este}

\paragraph{Veľmi dôležitá poznámka.}
Niekedy je potrebné nadpisom označiť odsek. Text pokračuje hneď za nadpisom.



\section{Dôležitá časť} \label{dolezita}




\section{Ešte dôležitejšia časť} \label{dolezitejsia}




\section{Záver} \label{zaver} % prípadne iný variant názvu



%\acknowledgement{Ak niekomu chcete poďakovať\ldots}


% týmto sa generuje zoznam literatúry z obsahu súboru literatura.bib podľa toho, na čo sa v článku odkazujete
\bibliography{literatura}
\bibliographystyle{abbrv} % prípadne alpha, abbrv alebo hociktorý iný
\end{document}
